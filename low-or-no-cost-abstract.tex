\documentclass{svmult}

\usepackage{mathptmx,helvet,courier,graphicx,multicol}

\begin{document}

\title*{Low or no cost distributed evolutionary computation}
\author{Juan J. Merelo-Guervós}
\institute{GeNeura Group (\url{http://geneura.wordpress.com}), Department of Computer Architecture (\url{http://atc.ugr.es}) and CITIC (\url{http://citic.ugr.es}), University of Granada (Spain) (\url{http://www.ugr.es})}


\maketitle

\abstract{From the era of big science we are back to the "do it yourself", where you don't have any money to buy clusters or subscribe to grids but still have algorithms that crave many computing nodes and need them to measure scalability. Fortunately, this coincides with the era of big data, cloud computing, and browsers that include JavaScript virtual machines. Those are the reason why this talk will focus on two different aspects of volunteer or freeriding computing: first, the pragmatic: where to find those resources, which ones can be used, what kind of support you have to give them; and then, the theoretical: how evolutionary algorithms can be adapted to a environment in which nodes come and go, have different computing capabilites and operate in complete asynchrony of each other.}


\end{document}